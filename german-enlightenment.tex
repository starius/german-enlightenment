\documentclass{beamer}

\usepackage[utf8]{inputenc}
\usepackage[russian]{babel}
\usepackage{latexsym}
\usepackage{hyperref}

\title[Немецкое просвещение]{Немецкое просвещение: идеи и представители}
\date{2012}
\usetheme{Warsaw}
\usecolortheme{seahorse}
\setbeamertemplate{footline}[frame number]{}
\setbeamertemplate{navigation symbols}{}

\begin{document}
    \begin{frame}
        \titlepage
    \end{frame}

    \begin{frame}{Характеристики}
        \begin{itemize}
        \item метод рационального анализа (Вольф) $\rightarrow$ Кант
        \item религиозные проблемы
        \item эстетика
        \end{itemize}
        Метод Вольфа фиксирует, по Канту,
        <<надежный путь науки путем регулярного определения принципов,
        педантичного уточнения понятий, утонченной строгости доказательств,
        отказа от дерзких шагов в выводах>>.
    \end{frame}

    \begin{frame}{Истоки}
        \begin{itemize}
        \item Лейбниц
        \item научные теории Нтютона $\rightarrow$ Кант
        \item Спиноза
        \item английские и французские просветители
        \end{itemize}
    \end{frame}

    \begin{frame}{Ссылки}
        \begin{itemize}
        \item Д. Антисери, Дж. Реале.
            Западная философия от истоков до наших дней
        \end{itemize}
    \end{frame}

\end{document}

