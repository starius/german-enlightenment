\documentclass{beamer}

\usepackage[utf8]{inputenc}
\usepackage[russian]{babel}
\usepackage{latexsym}
\usepackage{hyperref}

\title[Немецкое просвещение]{Немецкое просвещение: идеи и представители}
\date{2012}
\usetheme{Warsaw}
\usecolortheme{seahorse}
\setbeamertemplate{footline}[frame number]{}
\setbeamertemplate{navigation symbols}{}

\begin{document}
    \begin{frame}
        \titlepage
    \end{frame}

    \begin{frame}{Характеристики}
        \begin{itemize}
        \item метод рационального анализа (Вольф) $\rightarrow$ Кант
        \item религиозные проблемы
        \item эстетика
        \end{itemize}
        Метод Вольфа фиксирует, по Канту,
        <<надежный путь науки путем регулярного определения принципов,
        педантичного уточнения понятий, утонченной строгости доказательств,
        отказа от дерзких шагов в выводах>>.
    \end{frame}

    \begin{frame}{Истоки}
        \begin{itemize}
        \item Лейбниц
        \item научные теории Нтютона $\rightarrow$ Кант
        \item Спиноза
        \item английские и французские просветители
        \end{itemize}
    \end{frame}

    \begin{frame}{Предшественники Просвещения в Германии}
        \begin{itemize}
        \item Чирнхауз
        \item Пуфендорф
        \item Томазий
        \end{itemize}
    \end{frame}

    \begin{frame}{Предшественники Просвещения в Германии}
            {Э. В. фон Чирнхауз: искусство истины}
        <<Исцеление ума, или Общие наставления в искусстве открытия>>.
        На базе математической модели предлагает <<искусство истины>>.

        Очевидные истины, на базе которых формируется знание:
        \begin{enumerate}
        \item мы осознаем многие веши - что нам нравится и что не нравится
            (отсюда понятия добра и зла и основы этики);
        \item некоторые вещи доступны нашему восприятию, другие - нет;
        \item с помощью внутренних и внешних чувств
            мы создаем образ внешних объектов.
        \end{enumerate}
        Опыт -- внутреннее озарение $\leftarrow$ Декарт
    \end{frame}

    \begin{frame}{Предшественники Просвещения в Германии}
            {Самюэль Пуфендорф: естественное право и проблема разума}
        \begin{itemize}
        \item труд <<О естественном и международном праве>>
        \item Естественное право это разумное право, поэтому оно не может
            основываться на религиях, различных у разных народов.
        \item Был убежден, что на этой основе можно создать науку о праве,
            такую же точную, как физика.
        \end{itemize}
    \end{frame}

    \begin{frame}{Предшественники Просвещения в Германии}
            {Христиан Томазий: различие между правом и моралью}
        К началу XVIII столетия, под влиянием Локка и сенсуалистов,
        Томазий склонился к идеям Просветительства.

        Человеческий разум, а не Откровение, стал критерием истины во всех
        видах человеческой деятельности и, следовательно, юридических норм.

        Различение и определение самостоятельной категории юрисдикции:
        \begin{enumerate}
        \item юридическое понятие -- justum (справедливое),
        \item моральное -- honestum (честное),
        \item социальное -- decorum (приличное)
        \end{enumerate}
        Интерсубъективность и принуждение -- характеристики justum.

        Свобода мысли и религии.

        $\rightarrow$ Кант.
    \end{frame}

    \begin{frame}{Ссылки}
        \begin{itemize}
        \item Д. Антисери, Дж. Реале.
            Западная философия от истоков до наших дней
        \end{itemize}
    \end{frame}

\end{document}

