\documentclass{beamer}

\usepackage[utf8]{inputenc}
\usepackage[russian]{babel}
\usepackage{latexsym}
\usepackage{hyperref}

\title[Немецкое просвещение]{Немецкое просвещение: идеи и представители}
\date{2012}
\usetheme{Warsaw}
\usecolortheme{seahorse}
\setbeamertemplate{footline}[frame number]{}
\setbeamertemplate{navigation symbols}{}

\begin{document}
    \begin{frame}
        \titlepage
    \end{frame}

    \begin{frame}{Характеристики}
        \begin{itemize}
        \item метод рационального анализа (Вольф) $\rightarrow$ Кант
        \item религиозные проблемы
        \item эстетика
        \end{itemize}
        Метод Вольфа фиксирует, по Канту,
        <<надежный путь науки путем регулярного определения принципов,
        педантичного уточнения понятий, утонченной строгости доказательств,
        отказа от дерзких шагов в выводах>>.

        Лекции начинают читать на немецком, а не на латыни (Томазий, Вольф).
    \end{frame}

    \begin{frame}{Истоки}
        \begin{itemize}
        \item Лейбниц
        \item научные теории Ньютона $\rightarrow$ Кант
        \item Спиноза
        \item английские и французские просветители
        \end{itemize}
    \end{frame}

    \begin{frame}{Предшественники Просвещения в Германии}
        \begin{itemize}
        \item Чирнхауз
        \item Пуфендорф
        \item Томазий
        \end{itemize}
    \end{frame}

    \begin{frame}{Предшественники Просвещения в Германии}
            {Э. В. фон Чирнхауз: искусство истины}
        <<Исцеление ума, или Общие наставления в искусстве открытия>>.
        На базе математической модели предлагает <<искусство истины>>.

        Очевидные истины, на базе которых формируется знание:
        \begin{enumerate}
        \item мы осознаем многие вещи - что нам нравится и что не нравится
            (отсюда понятия добра и зла и основы этики);
        \item некоторые вещи доступны нашему восприятию, другие - нет;
        \item с помощью внутренних и внешних чувств
            мы создаем образ внешних объектов.
        \end{enumerate}
        Опыт -- внутреннее озарение $\leftarrow$ Декарт
    \end{frame}

    \begin{frame}{Предшественники Просвещения в Германии}
            {Самюэль Пуфендорф: естественное право и проблема разума}
        \begin{itemize}
        \item труд <<О естественном и международном праве>>
        \item Естественное право это разумное право, поэтому оно не может
            основываться на религиях, различных у разных народов.
        \item Был убежден, что на этой основе можно создать науку о праве,
            такую же точную, как физика.
        \end{itemize}
    \end{frame}

    \begin{frame}{Предшественники Просвещения в Германии}
            {Христиан Томазий: различие между правом и моралью}
        К началу XVIII столетия, под влиянием Локка и сенсуалистов,
        Томазий склонился к идеям Просветительства.

        Человеческий разум, а не Откровение, стал критерием истины во всех
        видах человеческой деятельности и, следовательно, юридических норм.

        Различение и определение самостоятельной категории юрисдикции:
        \begin{enumerate}
        \item юридическое понятие -- justum (справедливое),
        \item моральное -- honestum (честное),
        \item социальное -- decorum (приличное)
        \end{enumerate}
        Интерсубъективность и принуждение -- характеристики justum.

        Свобода мысли и религии.

        $\rightarrow$ Кант.
    \end{frame}

    \begin{frame}{Пиетизм и его связи с Просвещением}
        Пиетизм придает особой значимости личному благочестию,
        религиозным переживаниям верующих, ощущению живого общения с Богом.

        В XVII веке противопоставлялся догматичной лютеранской ортодоксии.

        Задачи:
        \begin{enumerate}
        \item полемика с догматами господствующего ортодоксального лютеранства;
        \item отстаивание свободы личности;
        \item основа практической веры перед схоластической теологией.
        \end{enumerate}

        Пиетизм связан с философами Просвещения:
        \begin{itemize}
        \item Томазий
        \item Вольф (пострадал)
        \end{itemize}
    \end{frame}

    \begin{frame}{Христиан Вольф}
        \begin{itemize}
        \item Автор многих математических трудов.
            Из логики Лейбница Вольф устраняет важные логико-формальные аспекты,
            сводя ее к силлогистике.
            Это направление станет господствующим в немецком Просвещении.
        \item Учебник <<Разумные мысли о силах человеческого рассудка и
            их исправном употреблении в познании истины>> заложил
            \emph{философскую терминологию}, которая сохранилась
            вплоть до наших дней.
        \item Вольф: <<В философском методе не следует применять термины
            без четкого определения и считать верным то,
            что не было убедительно доказано;
            в высказываниях нужно с одинаковым тщанием определять как субъект,
            так и предикат, с помощью которых объясняется то,
            что следует за ними>>.
        \end{itemize}
    \end{frame}

    \begin{frame}{Христиан Вольф}{Энциклопедия знания}
        Основываясь на философии Декарта и, главным образом, Лейбница,
        Вольф создает настоящую энциклопедию знания:
        \begin{itemize}
        \item теоретические науки: онтология, космология,
            рациональная психология, естественная теология;
        \item практические науки: практическая философия и естественное право,
            политика, экономика;
        \item эмпирико-теоретические науки: эмпирическая психология,
            телеология, догматическая физика;
        \item практико-эмпирические науки: технические дисциплины,
            экспериментальная физика.
        \end{itemize}
        Логика предваряет всю систему наук.
        Она опирается на два принципа -- непротиворечия в теории и
        достаточного основания для эмпирических наук.
    \end{frame}

    \begin{frame}{Христиан Вольф}
        \begin{itemize}
        \item Вольф придерживается космологического доказательства
            (Бог -- высший разум, оправдывающий порядок мира).
            Отклоняет онтологическое и телеологическое доказательства.
        \item Вслед за Лейбницем, считает, что отношения между душой и телом
            регулируются заранее установленной гармонией.
        \item Вольф считает этические нормы независимыми от
            любых теологических соображений (расходится с Лейбницем).
        \end{itemize}
    \end{frame}

    \begin{frame}{Александр Баумгартен}
        \begin{itemize}
        \item Метафизика -- наука о первых принципах человеческого познания.
        \item Работу Баумгартена <<Метафизика>>, вобравшую в себя Вольфа,
            считают базовым текстом для университетских занятий.
        \item По оценке Канта, Баумгартен заложил философские основы эстетики
            (термин введен Баумгартеном).\\
            Эстетика занимается чувственным феноменом,
            не делая попыток перейти от него к <<причинам>>,
            так как это подавило бы переживания.
        \end{itemize}
    \end{frame}

    \begin{frame}{Герман Самюэль Реймарус}
            {Натуральная религия против религии откровения}
        \begin{itemize}
        \item Считал, что существование Бога, Создателя мира,
            реальность провидения и бессмертие души доказуемы.
        \item Критикует французских материалистов, поскольку без религии
            нет морали и надежды на счастливую жизнь.
        \item Считал, что если библейская религия выступает против
            естественной религии, значит, она ложна.
        \end{itemize}
    \end{frame}

    \begin{frame}{Мозес Мендельсон}
            {Различие между религией и государством}
        \begin{itemize}
        \item Издатель курса популярной философии.
        \item Религиозная унификация невозможна, поскольку такая унификация
            потребовала бы юридической формулировки, апеллирующей к власти.
            Но религия и государство различны по сути:
            <<Государство обязывает и вынуждает, религия учит и убеждает>>.
        \end{itemize}
    \end{frame}

    \begin{frame}{Ссылки}
        \begin{itemize}
        \item Д. Антисери, Дж. Реале.
            Западная философия от истоков до наших дней
        \end{itemize}
    \end{frame}

\end{document}

